%-------------------------
% Resume in Latex
% Author : Jake Gutierrez
% Based off of: https://github.com/sb2nov/resume
% License : MIT
%------------------------
\documentclass[letterpaper,11pt]{article}
\usepackage[utf8]{inputenc}
\usepackage[T2A]{fontenc}
\usepackage[russian,english]{babel}
\usepackage{latexsym}
\usepackage[empty]{fullpage}
\usepackage{titlesec}
\usepackage{marvosym}
\usepackage[usenames,dvipsnames]{color}
\usepackage{verbatim}
\usepackage{enumitem}
\usepackage[hidelinks]{hyperref}
\usepackage{fancyhdr}
\usepackage{tabularx}
\usepackage{outlines}
\usepackage{ifthen}
\usepackage{xltabular}
\usepackage{graphicx}
\usepackage{fontawesome5} % icons
\input{glyphtounicode}
\usepackage[export]{adjustbox}
\usepackage{geometry}
\geometry{
  a4paper,
  top=20mm, 
  right=20mm, 
  bottom=15mm, 
  left=20mm
}

\pagestyle{fancy}
\fancyhf{} % clear all header and footer fields
\fancyfoot{}
\renewcommand{\headrulewidth}{0pt}
\renewcommand{\footrulewidth}{0pt}

% Adjust margins
\addtolength{\oddsidemargin}{-0.5in}
\addtolength{\evensidemargin}{-0.5in}
\addtolength{\textwidth}{1in}
\addtolength{\topmargin}{-.5in}
\addtolength{\textheight}{1.0in}

\urlstyle{same}

\raggedbottom
\raggedright
\setlength{\tabcolsep}{0in}

% Sections formatting
\titleformat{\section}{
  \vspace{-4pt}\scshape\raggedright\large
}{}{0em}{}[\color{black}\titlerule \vspace{-5pt}]

% Ensure that generate pdf is machine readable/ATS parsable
\pdfgentounicode=1

%-------------------------
% Custom commands
\newcommand{\resumeItem}[2]{
  \item\small{
    \textbf{#1}{: #2 \vspace{-2pt}}
  }
}

\newcommand{\resumeItemNoBold}[1]{
  \item\small{
    {#1 \vspace{-2pt}}
  }
}

% Just in case someone needs a heading that does not need to be in a list
\newcommand{\resumeHeading}[4]{
    \begin{tabular*}{0.99\textwidth}[t]{l@{\extracolsep{\fill}}r}
      \textbf{#1} & #2 \\
      \textit{\small#3} & \textit{\small #4} \\
    \end{tabular*}\vspace{-5pt}
}

\newcommand{\resumeSubheading}[4]{
  \vspace{-1pt}\item
    \begin{tabular*}{0.97\textwidth}[t]{l@{\extracolsep{\fill}}r}
      \textbf{#1} & #2 \\
      \textit{\small#3} & \textit{\small #4} \\
    \end{tabular*}\vspace{-5pt}
}

\newcommand{\resumeSubSubheadingItem}[3]{
  \vspace{-1pt}\item
    \begin{tabular*}{0.97\textwidth}{l@{\extracolsep{\fill}}r}
      \textbf{#1} | {#2} & \textit{\small #3} \\
    \end{tabular*}\vspace{-5pt}
}


\newcommand{\resumeSubSubheading}[2]{
    \begin{tabular*}{0.97\textwidth}{l@{\extracolsep{\fill}}r}
      \textit{\small#1} & \textit{\small #2} \\
    \end{tabular*}\vspace{-5pt}
}

\newcommand{\resumeSubItem}[2]{\resumeItem{#1}{#2}\vspace{-4pt}}

\renewcommand{\labelitemii}{$\circ$}

\newcommand{\resumeSubHeadingListStart}{\begin{itemize}[leftmargin=*]}
\newcommand{\resumeSubHeadingListEnd}{\end{itemize}}
\newcommand{\resumeItemListStart}{\begin{itemize}}
\newcommand{\resumeItemListEnd}{\end{itemize}\vspace{-5pt}}

\newcommand\clink[1]{{\usefont{T1}{lmtt}{m}{n} #1 }}

%-------------------------------------------
%%%%%%  CV STARTS HERE  %%%%%%%%%%%%%%%%%%%%%%%%%%%%


\begin{document}

%----------HEADING-----------------
\begin{tabularx}{\linewidth}{@{}m{0.8\textwidth} m{0.2\textwidth}@{}}
  {
      \huge{\textbf{Алексей Лапин}} \newline
      \small{
          \clink{
              \href{mailto:a.lapin03@gmail.com}{\faEnvelope~\underline{a.lapin03@gmail.com}} \textbf{ | }
              {\fontdimen2\font=0.75ex \href{tel:+79217770608}{\faPhone~\underline{+7 (921) 777-06-08}}}\textbf{ | }
              \href{https://github.com/AaLexUser}{\faGithub~\underline{github.com/AaLexUser}}
          } 
          \newline
          Санкт-Петербург, Россия
      }
  } & 
  {
      \hfill
      \includegraphics[width=2.8cm, frame]{../photo/me.png}
  }
\end{tabularx}
\vspace{-25pt}

%-----------EDUCATION-----------------
\section{Образование}
  \resumeSubHeadingListStart
    \resumeSubheading
      {\href{https://itmo.ru/}{Университет ИТМО}}{Санкт-Петербург, Россия}
      {\href{https://abit.itmo.ru/program/bachelor/system_software}{Бакалавриат: Системное и прикладное программное обеспечение 09.03.04}}{Сентябрь 2021 -- Июль 2025}
      
  \resumeSubHeadingListEnd

%-----------PROJECTS-----------------
\section{Проекты}
\resumeSubHeadingListStart

\resumeSubheading{
    \href{https://github.com/AaLexUser/ITMO-2-course/tree/main/Computational-mathematics/compl-math-lab-4}
    {Function approximation program}}{\textit{Апрель -- Май 2023}}{С++, SFML}{ }
    \resumeItemListStart
      \resumeItemNoBold{Программа находит функцию,  являющуюся наилучшим приближением заданной табличной функции по методу наименьших квадратов.}
      \resumeItemNoBold{Достигнуто интерактивное взаимодействие с графиком (перемещение, масштабирование, добавление графиков на одно окно), путем написания библиотеки на основе \textbf{SFML}.}
    \resumeItemListEnd

    \resumeSubheading{
    \href{https://github.com/AaLexUser/ITMO-2-course/tree/main/Programming-languages/assignment-5-sepia-filter}
    {Фильтр сепия на C и Ассемблере}}{\textit{Январь 2023}}{C, x86\_64asm} { }
    \resumeItemListStart
      \resumeItemNoBold{Достигнуто ускорение выполнения программы на ассемблере в 10 раз по сравнению с реализацией на C, путем использования \textbf{векторных инструкций процессора SSE}.}
      \resumeItemNoBold{Реализована библиотека для работы с \textbf{BMP} файлами картинок.}
    \resumeItemListEnd

    \resumeSubheading{
    \href{https://github.com/AaLexUser/ITMO-2-course/tree/main/Web-programming/web-4}
    {Point Application}}{\textit{Декабрь -- Февраль 2023}}{Java, TypeScript, React, Redux, Spring Boot, Oracle}{ }
    \resumeItemListStart
      \resumeItemNoBold{Динамический программный проект, предназначенный для создания точек на графике и управления ими.}
      \resumeItemNoBold{Разработал \textbf{RESTful API}, использующий архитектуру \textbf{Spring Boot} для обработки запросов и ответов.}
      \resumeItemNoBold{Аутентификация на основе \textbf{JWT-токена} с использованием \textbf{Spring Security}.}
      \resumeItemNoBold{Использовал \textbf{Spring Data} для взаимодействия с \textbf{СУБД Oracle}, сопоставляя объекты Java с таблицами.}
    \resumeItemListEnd

\resumeSubheading{
    \href{https://github.com/AaLexUser/Route-manager-application}
    {Route Manager}}{\textit{Февраль -- Октябрь 2022}}{Java Core, Stream API, Reflection API, PostgreSQL, SLF4J}{ }
    \resumeItemListStart
      \resumeItemNoBold{Клиент-серверное приложение предназначенное для хранения маршрутов пользователей.}
      \resumeItemNoBold{Использовались \textbf{Java Collections} для эффективного хранения данных.}
      \resumeItemNoBold{Использовал \textbf{Java’s Stream API} и лямбда-выражения для эффективной обработки данных.}
      \resumeItemNoBold{Реализован обмен данными между клиентом и сервером по протоколу \textbf{TCP}.}
      \resumeItemNoBold{На серевере реализовано \textbf{многопоточное} чтение запросов, обработка и отправка ответов.}
      \resumeItemNoBold{Для работы с запросами от нескольких пользователей применяется \textbf{Java NIO Selector}.}
      \resumeItemNoBold{Для хранения данных используется \textbf{СУБД PostgreSQL}.}
      \resumeItemNoBold{Реализована собственная библиотека для автоматичесческого внедрения зависимостей \textbf{DI (Dependency injection)} с использованием \textbf{Java Reflection API}.}
      \resumeItemNoBold{Реализовано логирование событий с помощью библиотеки \textbf{slf4j}.}
      \resumeItemNoBold{Реализовано хеширование паролей алгоритмом \textbf{SHA-256}.}


    \resumeItemListEnd

\resumeSubHeadingListEnd 

\section{Награды}
\resumeSubHeadingListStart
  \resumeSubheading{
    \href{https://ftc-events.firstinspires.org/2019/team/12529\#awards}
    {First Russia Robotics Championship}}{Красноярск, Россия}
    {Победа в номинации Think Award Winner в составе команды KTM.}{Февраль 2020}
    
\resumeSubHeadingListEnd


%--------PROGRAMMING SKILLS------------
\section{Технические навыки}
 \resumeSubHeadingListStart
   \resumeSubItem{Языки Программирования}{C, C++, x86\_64asm, Java, Python, SQL (PostgreSQL), JavaScript, HTML/CSS}
   \resumeSubItem{Фреймворки}{Spring Boot, Spring, Spring Data, Spring Security}
   \resumeSubItem{Библиотеки}{React, Pandas, NumPy, Matplotlib, SFML}
   \resumeSubItem{Инструменты разработчика}{Git, GitHub Actions, Gradle, Maven, CLion, PyCharm, IntelliJ, Webstorm, Jupyter Notebook, GoogleCollab}
   \resumeSubItem{Операционные системы}{Linux, UNIX}
 \resumeSubHeadingListEnd
 \section{Дополнительная информация}
 \resumeSubHeadingListStart
   \resumeSubItem{Языки}{Английский язык -- B1}
 \resumeSubHeadingListEnd
\end{document}
